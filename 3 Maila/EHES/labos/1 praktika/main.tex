\documentclass{article}

%%\chapterstyle{veelo}
%% KAPITULU FORMATU BATZUK ARAZOAK EMATEN DITUZTE DOKUMENTUA 
%% MOZTERAKOAN (croppedsize AUKERAREKIN, BEHEAN AGERTZEN DENA). 
%% JARRAIAN ALTERNATIBA SGURU BATZUK DAUDE

%\chapterstyle{default}
%\chapterstyle{bianchi}
%\chapterstyle{brotherton}
%\chapterstyle{demo2}
%\chapterstyle{ell}
%\chapterstyle{ger}
%\chapterstyle{wilsondob}






%% PAKETE HONEN AGINDUAK ERABILTZEN BADIRA, ALDATU HIZKUNTZA AUKERA %%
%% SI SE USAN LOS COMANDOS DE ESTE PAQUETE, CAMBIAR LA OPCIÓN DE IDIOMA %%
\usepackage{cancel}
\usepackage{amsmath}
\usepackage[utf8]{inputenc}
\usepackage{amssymb}
\usepackage{tikz}
\usetikzlibrary{positioning}
\usetikzlibrary{arrows.meta}
\usepackage{graphicx}
\usepackage[linkcolor=black, citecolor=black, urlcolor=black]{hyperref}
\usepackage[letterpaper,top=2cm,bottom=2cm,left=3cm,right=3cm,marginparwidth=1.75cm]{geometry}
%\usepackage{enumitem}

%% INFORMAZIOA / INFORMACIÓN %%
\newcommand{\egilea}{
    Urko Bidaurre\\
}
\newcommand{\irakaslea} {
      Alicia Perez Ramirez
}
\newcommand{\izenburua}{Datuen deskribapen operatiboa}
\newcommand{\data}{\today}


%% BAKARRIK BAT DESKOMENTATU, GRADU AMAIERAKO LANA BADA. BESTELA, DENAK KOMENTATU%%
%% DESCOMENTAR SOLO UNO SI ES UN TRABAJO FIN DE GRADO. SINO, COMENTAR TODOS%%

\newcommand{\ikasketak}{Kudeaketaren eta Informazio Sistemen Informatikaren Ingeniaritzako Gradua}
%\newcommand{\ikasketak}{Grado en Ingeniería Informática}
%\newcommand{\ikasketak}{Adimen Artifizialeko Gradua}
%\newcommand{\ikasketak}{Grado en Inteligencia Artificial}


%% BAKARRIK BAT DESKOMENTATU, INGENIARITZA INFORMATIKAKOA BADA. BESTELA, DENAK KOMENTATU %%
%% DESCOMENTAR SOLO UNO SI ES EN INGENIERÍA INFORMÁTICA. SINO, COMENTAR TODOS %%

%\newcommand{\espezialitatea}{Konputazioa}
%\newcommand{\espezialitatea}{Computación}
%\newcommand{\espezialitatea}{Software Ingeniaritza}
%\newcommand{\espezialitatea}{Ingeniería de Software}
%\newcommand{\espezialitatea}{Konputagailuen Ingeniaritza}
\newcommand{\espezialitatea}{
    Erabakiak Hartzeko Euskarri Sistemak\\
    }


%% BAKARRIK BAT DESKOMENTATU MASTER AMAIERAKO LANA BADA. BESTELA, DENAK KOMENTATU%%
%% DESCOMENTAR SOLO UNO SI ES UN TRABAJO FIN DE MASTER. SINO, COMENTAR TODOS%%

%\newcommand{\ikasketak}{Máster Universitario en Ingeniería Computacional y Sistemas Inteligentes}
%\newcommand{\ikasketak}{Konputazio Ingeniaritza eta Sistema Adimentsuak Unibertsitate Masterra}

%% TESIENTZAKO KONFIGURATU
%\newcommand{\saila}{Department of Computer Science and Artificial Intelligence}
%\newcommand{\phdarloa}{Computer Science}
%% KOMENTATU A4 FORMATUAN ATERATZEKO, FORMATU TXIKIAGOAN AGERTZEKO 
%% GERO INPRIMATZEKO. MOZTEKO MARKAK IKUSTEKO GEHITU showtrims AUKERA
%% memoir KLASEAREN ARGUMENTUETAN
%\newcommand{\croppedsize}

%% DOKUMENTUAREN HIZKUNTZA / IDIOMA DEL DOCUMENTO %%
%% BAKARRIK BAT DESKOMENTATU / DESCOMENTAR SOLO UNO %%
\newcommand{\euskaraz}
%\newcommand{\castellano}
%\newcommand{\english}


\input config/macros

\title{\izenburua}
\author{\egilea}
\date{\data}

\begin{document}
\thispagestyle{empty}

\newcommand{\HRule}{\rule{\linewidth}{0.5mm}} 

\thispagestyle{empty}
\begin{center}
	
  \includegraphics[width=0.75\textwidth]{config/FacultadInformatica-Gipuzkoa-bilingue-positivo-alta.jpg} \\[2cm]
  
  
  {\LARGE {\gapizenburua}}\\[0.5cm]
  {\Large \ikasketak}\\[0.25cm]
  {\espezialitatea}
  \vspace{2cm}
  
  
  \HRule\\[0.5cm]
  {\LARGE 
    \textbf{\izenburua}
  }
  \HRule\\[0.5cm]

  {\Large \textsl{\egilea}
  
   
  \vfill
  
  \textbf{\zuzendariaktestua}\\
  \irakaslea\\[2cm]
  
  \data
  }
  
\end{center}
\newpage
\tableofcontents
\newpage
\section{Materiala}
\begin{itemize}
  \item Weka aplikazioa
  \item Baliabide bibliografikoak:
  \begin{itemize}
    \item Informazio orokorra adibideekin: [Witten et al., 2011, Chap. 2]
    \item Kontsulta praktikoak: \url{https://waikato.github.io/weka-wiki/}
  \end{itemize}
  \item eGelatik eskuragarri:
  \begin{itemize}
    \item Baliabide orokorrak: aplikazioaren eskuliburua
    \item Praktikarako datu-sorta: heart-c.arff
  \end{itemize}
\end{itemize}

\section{Helburuak}
Praktika honen helburuak datu meatzaritzarako ikuspegi orokorra ematea da Weka aplikazioaren
bitartez. Honetarako datu meatzaritzan informazioa erauzteko hiru teknika nagusiak aipatuko
dira: \textbf{iragarpena}, \textbf{clustering} eta \textbf{asoziazioa}. Wekarako sarrera gisa ARFF fitxategien kudeaketan sakonduko dugu iragarpen ataza baten bitartez.

Hurrengo konpetentziak landu:
\begin{itemize}
  \item \textbf{Zeharkako konpetentziak}:
  \begin{itemize}
    \item Lan autonomoa
    \item Pentsamendu kritikoa
  \end{itemize}
  \item \textbf{Konpetentzia orokorrak:}
  \begin{itemize}
    \item Ikasketa automatikoaren funtsa deskribatzeko gai izatea
    \item Datuen deskribapen operatiboa emateko gai izatea
    \item Wekarako sarrera: atal ezberdinak bereizteko gai izatea
  \end{itemize}
\end{itemize}

\section{Gidoia}
\subsection{Aldez aurretiko lana}
Praktika hau egiten hasi aurretik honako lanal eskatzen dira: 
\begin{enumerate}
  \item Gai hauei buruzko informazioa irakurri
  \begin{itemize}
    \item Machine learning: datuetatik ezaguerara . Ikasketa automatikoaren funtsa, datuetatik
          erabiliz ezaguera edo informazioa erauztea da. Datuek, lortu nahi den ezagueraren
          adierazgarri izan behar dute. Lagin-espazioko adibide esanguratsuak.
          Irakurri: \cite[Chap. 1]{Witten2011}
    \item Weka-ko datuen formatua: ARFF. Atributuak erabiltzen dira datuen deskribapen
          operatiboa emateko. Izan ere, atributuen bitartez deskribatutako datuei buruketako
          instantzia (edo adibide) deritze. Alegia, instantziak karakterizatzeko atributuak erabiltzen dira.
          Irakurri: \cite[Chap. 2]{Witten2011}
  \end{itemize}
  \item Weka deskargatu eta instalatu: \url{http://www.cs.waikato.ac.nz/ml/weka/}
\end{enumerate}
\subsection{Datu meatzaritzako paradigmak}
Datu meatzaritzak mota honetako atazak ebazteko balio du:
\begin{itemize}
  \item Gene batzuen presentziaren arabera, etorkizunean gaixotasun bat izateko probabilitatea
eman.
  \item Biometria: begiko irisaren ezaugarri batzuen arabera, pertsona identifikatu
  \item Bezero baten erosketen arabera, beste produktu batzuk gomendatui
  \item Espezie bateko ezaugarrien arabera, bariedadeak bereiztu, alegia, taxonomiak deskubritu
  \item Aseguru etxeetan antzeko jokaerak dituzten bezeroei antzeko produktuak eskaini
  \item Iraganean entzundako musikaren arabera, musika gomendatu
\end{itemize}
Datuetatik informazioa erauzteko hiru paradigma nagusi bereizgten dira: iragarpena (edo
sailkapen gainbegiratua), clustering (sailkapen ez-gainbegiratua) eta asoziazioa. Aurreko atazak
hauetako batean sartzen dira. Hiru paradigmak deskribatu eta bakoitzerako adibideak eman,
horretarako, iturri hau erabilgarria da: \cite[Sec. 2.1 y Sec. 1.3]{Witten2011}

\subsection{Datuen deskribapen operatiboa}
Praktika honetarako erabiliko dugun fitxategia \texttt{heart-c.arff} da.
\begin{enumerate}[\textbf{\arabic*.}]
  \item \textbf{Zein motatako informazioa (audio, irudiak, ...) dakar .arff fitxategiak? Zein da ARFFren esannahia? Zertarako erabiltzen dira mota honetako fitxategiak?} \cite[Sec. 2.1, 2.2, 11.1]{Witten2011}
  \begin{itemize}
    \item Informazioa: ARFF artxiboak ez du ez audiorik ez irudirik. Pazienteen ezaugarri klinikoak deskribatzen dituzten datu egituratu alfanumerikoak ditu.
    \item ARFFren esanahia: \emph{Attribute-Relation File Format}-en siglak. Wekaren jatorrizko formatua da, atributu-multzo bat partekatzen duten instantzien zerrenda deskribatzeko.
    \item Erabilera: Ikasketa automatikoko algoritmoek (Machine Learning) patroiak aurkitzeko prozesatuko dituzten datuak gordetzeko erabiltzen dira.
  \end{itemize}
  \item  \textbf{Editatu \texttt{.arff} fitxategia testu editore batekin. Burukoan agertzen den atazako deskribapena aztertu eta ondorengo galderei erantzun:}
  \begin{enumerate}
    \item Zertan datza ataza? Iragarpen (\emph{prediction}), taldekatze (\emph{clustering}) ala elkarketa (\emph{association}) buruketa da?

    Diagnostikoa adierazten duen \texttt{num} (bihotzeko gaixotasuna) izeneko azken atributua dagoenez, eta besteetan oinarrituta iragarri nahi dugunez, Iragarpen ataza bat dela esan dezakegu, eta zehazkiago, saikapen ataza bat (irteera etiketa bat delako, ez zenbaki jarraitu bat).
   
    \item Buruketako deskribapenaren arabera, zenbat balio har ditzake klaseak? Daukagun
lagin multzoan, zenbat balio har ditzake klaseak?

    \texttt{num} klaseak 5 balio posiblerekin definituta dago: '$<50$', '$>50_1$', '$>50_2$', '$>50_3$', '$>50_4$'.

    \item  \texttt{.arff} fitxategian  \emph{\%} ikurrarekin hasten diren lerroak, fitxategiko parte eragile dira?
    
    Ez, ez dira fitxategiko parte eragile. Datuak kargatzerakoan \emph{Weka}-k baztertzen dituen komentarioak dira.
  \end{enumerate}
  \item \textbf{Definitu: “Instantzia” eta “Atributu”} \cite[Sec. 2.2, 2.3]{Witten2011}
  \begin{itemize}
    \item \textbf{Instantzia:} Datu-errenkada bakoitza da (\texttt{@data}-ren ondoren agertzen diren errenkada bakoitza). Adibide zehatz bat adierazten du; kasu honetan, paziente espezifiko bat, bere datu klinikoekin.
    \item \textbf{Atributu:} Definitutako zutabe edo propietate bakoitza da (\texttt{@attribute} erabiliz). Pazienteak deskribatzen dituzten ezaugarriak irudikatzen dituzte, hala nola adina, kolesterola eta abar.
  \end{itemize}
\end{enumerate}
\newpage
\bibliographystyle{apalike}
\bibliography{sample}
\begin{itemize}
    \item \emph{https://www.lystloc.com/blog/what-is-a-travelling-salesman-problem-tsp/}
    \item \emph{SYMPLEX: SIMPLEX SOLVER}
\end{itemize}

\end{document}