%% KODIFIKAZIOA %%
% NOTA: inputenc ya se carga en main.tex con [utf8]; evitamos el choque de opciones.
%\usepackage[utf8x]{inputenc}
\usepackage[T1]{fontenc}
\usepackage{listings}
\usepackage{xcolor}

\definecolor{mygreen}{rgb}{0,0.6,0}
\definecolor{mygray}{rgb}{0.5,0.5,0.5}
\definecolor{myblue}{rgb}{0.2,0.2,0.6}
\definecolor{mymauve}{rgb}{0.58,0,0.82}

\lstset{
    language=Java,
    backgroundcolor=\color{white},  
    basicstyle=\ttfamily\footnotesize,  
    numbers=left,                 
    numberstyle=\tiny\color{mygray}, 
    keywordstyle=\color{blue},     
    commentstyle=\color{mygreen},  
    stringstyle=\color{mymauve},   
    breaklines=true,               
    captionpos=b,                  
    frame=single,                  
    rulecolor=\color{black},       
    showspaces=false,              
    showstringspaces=false,        
    showtabs=false,                
    tabsize=4,                     
}

\hypersetup{
    colorlinks=true,       % enlaces en color
    urlcolor=black,         % color para los enlaces
    linkcolor=black,       % color para enlaces internos
    citecolor=black,       % color para citas
    breaklinks=true,       % permite saltos de línea en URLs largas
    urlbordercolor={0 1 1} % Color de bordes para enlaces de URL
}

%% FONTS %%
\usepackage{libertine}
\usepackage{inconsolata}

% HYPERREFERENCES 
%\usepackage[hyperindex,bookmarks,colorlinks=true,citecolor=blue,urlcolor=blue,linkcolor=blue,pagebackref]{hyperref}
% \renewcommand*{\backref}[1]{}
% \renewcommand\backreftwosep{, }
% \renewcommand\backrefsep{, }

%% MISC. %%
\usepackage{appendix}
\usepackage{placeins}
\usepackage[figuresright]{rotating}
\usepackage{graphicx}
\usepackage{subfig}
\usepackage{float}
%\floatstyle{boxed}
\restylefloat{figure}
\usepackage{xcolor, graphicx}
\usepackage{multirow}
\usepackage{pdfpages}
\usepackage{xspace}
\usepackage{microtype}
\usepackage{longtable}
%\setsecnumdepth{subsubsection}
%\maxtocdepth {subsection}
\setlength{\parskip}{5pt}
\makeatletter
\renewcommand{\counterwithin}{\@ifstar{\@csinstar}{\@csin}}
\makeatother
\usepackage{doi}
\usepackage{amsmath, amssymb, mathrsfs, mathtools}
\usepackage[capitalise]{cleveref}


%\showtrimsoff

\ifdefined\euskaraz%
\newcommand{\upvehu}{Euskal Herriko Unibertsitatea UPV/EHU}
\newcommand{\gapizenburua}{Bigarren praktika}
\newcommand{\malizenburua}{Master Tesia}
\newcommand{\thesisizenburua}{Doktorego Tesia}
\newcommand{\informatikafakultatea}{Informatika Fakultatea}
\renewcommand{\abstract}{Laburpena}
\newcommand{\zuzendariaktestua}{Irakaslea}
\newcommand{\phdtext}{Dissertation submitted to the \saila of the University of the Basque Country (UPV/EHU) as partial fulfillment of the requirements for the Ph.D. degree in \phdarloa}
\fi
\ifdefined\castellano
\newcommand{\upvehu}{Universidad del País Vasco UPV/EHU}
\newcommand{\gapizenburua}{Trabajo de Fin de Grado}
\newcommand{\malizenburua}{Tesis de Máster}
\newcommand{\thesisizenburua}{Tesis Doctoral}
\newcommand{\informatikafakultatea}{Facultad de Informática}
\renewcommand{\abstract}{Resumen}
\newcommand{\zuzendariaktestua}{Dirección}
\newcommand{\phdtext}{Dissertation submitted to the \saila of the University of the Basque Country (UPV/EHU) as partial fulfillment of the requirements for the Ph.D. degree in \phdarloa}
\fi
\ifdefined\english
\newcommand{\upvehu}{University of the Basque Country UPV/EHU}
\newcommand{\gapizenburua}{Bachelor Thesis}
\newcommand{\malizenburua}{Master Thesis}
\newcommand{\thesisizenburua}{PhD Dissertation}
\newcommand{\informatikafakultatea}{Informatics Faculty}
\renewcommand{\abstract}{Abstract}
\newcommand{\zuzendariaktestua}{Advisors}
\newcommand{\phdtext}{Dissertation submitted to the \saila of the University of the Basque Country (UPV/EHU) as partial fulfillment of the requirements for the Ph.D. degree in \phdarloa}
\fi

\usepackage[font=small,labelfont=bf]{caption}

\ifdefined\euskaraz
\usepackage[basque]{babel}
% \hyphenation{Ko-man-do-in-ter-pre-ta-tzailea ba-te-ra-ga-rri-ta-suna ezau-garri}

\addto\captionsbasque{
	\renewcommand{\contentsname}{Gaien aurkibidea}
	\renewcommand{\listfigurename}{Irudien aurkibidea}
	\renewcommand{\listtablename}{Taulen aurkibidea}
	\renewcommand{\appendixname}{Eranskina}%
	\renewcommand{\appendixpagename}{Eranskinak}
	\renewcommand{\appendixtocname}{Eranskinak}
	\renewcommand{\bibname}{Bibliografia}
	\renewcommand{\tablename}{Taula}
	\renewcommand{\figurename}{Irudia}
}

% \renewcommand*{\backrefalt}[4]{%
% 	\ifcase #1%
% 	\or Ikusi #2 orrialdea.%
% 	\else Ikusi #2 orrialdeak.%
% 	\fi%
% }

%% Captionak euskarazko ordenean
\DeclareCaptionLabelFormat{euskaraz}{#2\bothIfSecond{\nobreakspace}{#1}}
\captionsetup{labelformat=euskaraz}
\fi

\ifdefined\castellano
\usepackage[spanish]{babel}
\addto\captionsspanish{
	\renewcommand{\contentsname}{Índice de contenidos}
	\renewcommand{\listfigurename}{Índice de figuras}
	\renewcommand{\listtablename}{Índice de tablas}
	\renewcommand{\appendixname}{Apéndice}%
	\renewcommand{\appendixpagename}{Apéndices}
	\renewcommand{\appendixtocname}{Apéndices}
	\renewcommand{\bibname}{Bibliografía}
	\renewcommand{\tablename}{Tabla}
	\renewcommand{\figurename}{Figura}
}
\renewcommand*{\backrefalt}[4]{%
\ifcase #1%
\or Ver página #2.%
\else Ver páginas #2.%
\fi%
}
\fi

\ifdefined\english
\usepackage[english]{babel}
\renewcommand*{\backrefalt}[4]{%
	\ifcase #1%
	\or See page #2.%
	\else See pages #2.%
	\fi%
}
\fi

\let\theoldbibliography\thebibliography
\renewcommand\thebibliography[1]{
	\theoldbibliography{#1}
	\setlength{\parskip}{0pt}
	\setlength{\itemsep}{4pt plus 0.3ex}
	\small
}

\ifdefined\croppedsize
\showtrimson
\trimLmarks
\settrimmedsize{240mm}{170mm}{*}
\settrims{28mm}{20mm}
\setlrmarginsandblock{30mm}{25mm}{*}
\setulmarginsandblock{30mm}{20mm}{*}

\checkandfixthelayout
\fi
