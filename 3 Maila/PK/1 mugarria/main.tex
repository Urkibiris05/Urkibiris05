\documentclass{article}
% chktex-file 44
% chktex-file 24
%%\chapterstyle{veelo}
%% KAPITULU FORMATU BATZUK ARAZOAK EMATEN DITUZTE DOKUMENTUA 
%% MOZTERAKOAN (croppedsize AUKERAREKIN, BEHEAN AGERTZEN DENA). 
%% JARRAIAN ALTERNATIBA SGURU BATZUK DAUDE

%\chapterstyle{default}
%\chapterstyle{bianchi}
%\chapterstyle{brotherton}
%\chapterstyle{demo2}
%\chapterstyle{ell}
%\chapterstyle{ger}
%\chapterstyle{wilsondob}






%% PAKETE HONEN AGINDUAK ERABILTZEN BADIRA, ALDATU HIZKUNTZA AUKERA %%
%% SI SE USAN LOS COMANDOS DE ESTE PAQUETE, CAMBIAR LA OPCIÓN DE IDIOMA %%
\usepackage{cancel}
\usepackage{amsmath}
\usepackage[utf8]{inputenc}
\usepackage{amssymb}
\usepackage{tikz}
\usepackage{makecell}
\usepackage{ragged2e}
\usetikzlibrary{positioning}
\usetikzlibrary{arrows.meta}
\usepackage{enumitem}
\usepackage{graphicx}
\usepackage{array}
\newcolumntype{C}[1]{>{\centering\arraybackslash}p{#1}}
\usepackage{float}
\usepackage[table]{xcolor}
\usepackage{colortbl}
\usepackage[linkcolor=black, citecolor=black, urlcolor=black]{hyperref}
\usepackage[letterpaper,top=2cm,bottom=2cm,left=3cm,right=3cm,marginparwidth=1.75cm]{geometry}

%% INFORMAZIOA / INFORMACIÓN %%
\newcommand{\egilea}{
    Urko Bidaurre\\
}
\newcommand{\irakaslea} {
      Ana Jesus Armendariz
}
\newcommand{\izenburua}{Bakarkako Proiektua}
\newcommand{\data}{\today}


%% BAKARRIK BAT DESKOMENTATU, GRADU AMAIERAKO LANA BADA. BESTELA, DENAK KOMENTATU%%
%% DESCOMENTAR SOLO UNO SI ES UN TRABAJO FIN DE GRADO. SINO, COMENTAR TODOS%%

\newcommand{\ikasketak}{Kudeaketaren eta Informazio Sistemen Informatikaren Ingeniaritzako Gradua}
%\newcommand{\ikasketak}{Grado en Ingeniería Informática}
%\newcommand{\ikasketak}{Adimen Artifizialeko Gradua}
%\newcommand{\ikasketak}{Grado en Inteligencia Artificial}


%% BAKARRIK BAT DESKOMENTATU, INGENIARITZA INFORMATIKAKOA BADA. BESTELA, DENAK KOMENTATU %%
%% DESCOMENTAR SOLO UNO SI ES EN INGENIERÍA INFORMÁTICA. SINO, COMENTAR TODOS %%

%\newcommand{\espezialitatea}{Konputazioa}
%\newcommand{\espezialitatea}{Computación}
%\newcommand{\espezialitatea}{Software Ingeniaritza}
%\newcommand{\espezialitatea}{Ingeniería de Software}
%\newcommand{\espezialitatea}{Konputagailuen Ingeniaritza}
\newcommand{\espezialitatea}{
    Proiektuen Kudeaketa\\
    }


%% BAKARRIK BAT DESKOMENTATU MASTER AMAIERAKO LANA BADA. BESTELA, DENAK KOMENTATU%%
%% DESCOMENTAR SOLO UNO SI ES UN TRABAJO FIN DE MASTER. SINO, COMENTAR TODOS%%

%\newcommand{\ikasketak}{Máster Universitario en Ingeniería Computacional y Sistemas Inteligentes}
%\newcommand{\ikasketak}{Konputazio Ingeniaritza eta Sistema Adimentsuak Unibertsitate Masterra}

%% TESIENTZAKO KONFIGURATU
%\newcommand{\saila}{Department of Computer Science and Artificial Intelligence}
%\newcommand{\phdarloa}{Computer Science}
%% KOMENTATU A4 FORMATUAN ATERATZEKO, FORMATU TXIKIAGOAN AGERTZEKO 
%% GERO INPRIMATZEKO. MOZTEKO MARKAK IKUSTEKO GEHITU showtrims AUKERA
%% memoir KLASEAREN ARGUMENTUETAN
%\newcommand{\croppedsize}

%% DOKUMENTUAREN HIZKUNTZA / IDIOMA DEL DOCUMENTO %%
%% BAKARRIK BAT DESKOMENTATU / DESCOMENTAR SOLO UNO %%
\newcommand{\euskaraz}
%\newcommand{\castellano}
%\newcommand{\english}


\input config/macros

\title{\izenburua}
\author{\egilea}
\date{\data}

\begin{document}
\thispagestyle{empty}

\newcommand{\HRule}{\rule{\linewidth}{0.5mm}} 

\thispagestyle{empty}
\begin{center}
	
  \includegraphics[width=0.75\textwidth]{config/FacultadInformatica-Gipuzkoa-bilingue-positivo-alta.jpg} \\[2cm]
  
  
  {\LARGE {\gapizenburua}}\\[0.5cm]
  {\Large \ikasketak}\\[0.25cm]
  {\espezialitatea}
  \vspace{2cm}
  
  
  \HRule\\[0.5cm]
  {\LARGE 
    \textbf{\izenburua}
  }
  \HRule\\[0.5cm]

  {\Large \textsl{\egilea}
  
   
  \vfill
  
  \textbf{\zuzendariaktestua}\\
  \irakaslea\\[2cm]
  
  \data}
  
\end{center}
\newpage
\tableofcontents
\newpage
\section{Lehen fasea: Helburuaren eta testuinguruaren analisia}

Atal honen helburu nagusia proiektuaren helmuga zehaztea eta hori lortzeko ingurunea aztertzea da. 
Proiektu honetan, ``Proiektuen Kudeaketa'' irakasgaian lortu nahi den kalifikazioa ezarriko da helburu gisa, 
eta lauhilekoan zehar eragina izango duten gainerako jarduerak identifikatuko dira denbora-kudeaketa egokia bermatzeko.

\subsection{Helburu akademikoa}
Proiektuaren arrakasta neurtzeko, helburu kuantitatibo eta zehatz bat definitu da. Helburu hau \textbf{SMART} irizpideen arabera (Espezifikoa, Neurgarria, Lor daitekeena, Esanguratsua eta Denboran mugatua) formulatu da, 
ikasturte amaieran lortu nahi den nota-tartea zehaztuz.

\begin{itemize}
  \item \textbf{Helburua:} 9 eta 10 bitartean.
\end{itemize}

\subsection{Beste jarduera eta konpromiso batzuk}
Helburu akademikoa lortzeko bidea ez da isolatuta gertatzen; ikaslearen errealitatea osatzen duten beste hainbat aldagaik baldintzatzen dute. 
Hori dela eta, jarraian, lauhilekoan zehar nire denbora eta arreta eskatuko duten jarduera guztiak identifikatu dira, bai oinarrizkoak (lo egin, jan, higienea, etab.),
bai akademikoak (klaseak, ikasketak, etab.) eta bai pertsonalak (kirola, lana, etab.). 
Analisi honek aukera ematen du benetako lan-karga dimentsionatzeko eta balizko denbora-gatazkak aurreikusteko. 

\begin{table}[ht]
\centering
\begin{tabular}{|l|c|}
  \hline 
  \textbf{Oinarrizko Jarduera} & \textbf{Asteko dedikazioa} \\ \hline
  Lo orduak & 52'5 ordu \\ \hline
  Jan orduak & 8 ordu \\ \hline
  Higienea & 2'5 ordu \\ \hline
  \textbf{GUZTIRA} & \textbf{63 ordu} \\ \hline
\end{tabular}
\caption{Astean zeharreko oinarrizko jarduerak, dagokien ordu kopuruarekin}%
\label{tab:oinarrizko_jarduerak}
\end{table}

\begin{table}[ht]
\centering
\begin{tabular}{|l|c|}
  \hline 
  \textbf{Jarduera} & \textbf{Asteko dedikazioa} \\ \hline
  Erabakiak Hartzeko Euskarri Sistemak, Klaseak & 4 ordu \\ \hline
  Erabakiak Hartzeko Euskarri Sistemak, Klasetik kanpo & 3 ordu \\ \hline
  Datu-baseen Kudeaketa, Klaseak & 4 ordu \\ \hline
  Datu-baseen Kudeaketa, Klasetik kanpo & 3 ordu \\ \hline
  Proiektuen Kudeaketa, Klaseak & 4 ordu \\ \hline
  Proiektuen Kudeaketa, Klasetik kanpo & 3 ordu \\ \hline
  Web Sistemak, Klaseak & 4 ordu \\ \hline
  Web Sistemak, Klasetik kanpo & 3 ordu \\ \hline
  Enpresak Kudeatzeko Softwarea, Klaseak & 4 ordu \\ \hline
  Enpresak Kudeatzeko Softwarea, Klasetik kanpo & 3 ordu \\ \hline
  Musika akademia & 1 ordu \\ \hline
  Musika, Klasetik kanpo & 1 ordu \\ \hline
  Kirol jarduera & 15 ordu \\ \hline
  Lan praktikak & 15 ordu \\ \hline
  Klaserako bidaiak & 5 ordu \\ \hline
  \textbf{GUZTIRA} & \textbf{72 ordu} \\ \hline
\end{tabular}
\caption{Astean zeharreko bestelako jarduera eta konpromisoak, dagokien ordu kopuruarekin}%
\label{tab:jarduerak_eta_konpromisoak}
\end{table}

\noindent Behin astean zeharreko jarduera bakoitzak zenbat denbora eramaten duen aztertuta, \textbf{astero batezbesteko 135 ordu} inbertitzen direla 
edonolako jardueretan ondorioztatzen da. Kontuan izanda aste batek 168 ordu dituela, 33 ordu geratuko lirateke soberan, aisialdian edo aurreikusi gabeko
bestelako jarduera edo konpromisoetan erabiltzeko (gertaera bereziak, irteerak, ezusteko lan-karga, etab.).
  
\newpage
\section{Bigarren fasea: Arriskuen identifikazioa eta analisia}

\noindent Behin helburua eta testuingurua argi izanda, atal honetan helburu hori betetzea oztopa dezaketen ziurgabetasun-egoerak aztertuko dira. 
Ez dira arazoak identifikatzen (jada gertatu edo ziur gertatuko diren gertaerak), baizik eta gerta daitezkeen eta helburuan eragin negatiboa edo positiboa izan dezaketen arriskuak.

\subsection{Arriskuen identifikazioa}

Arriskuak modu integralean identifikatzeko, hainbat kategoria hartu dira kontuan: denbora-kudeaketa, ikaskuntza-prozesua, osasuna, kanpo-faktoreak eta faktore pertsonalak. 
Prozesu honen bidez, proiektuaren garapenean eragina izan dezaketen gertaera potentzialen zerrenda osatu da:

\begin{itemize}
  \item \textbf{Ustekabeko lan-karga:} Bigarren lauhileko honetan, lan-praktika batzuekin hasiko nahiz, eta horrek denbora eskatuko du. 
  Nahiz eta teorian 15 ordu astero lan egin behar izan, bestelako arrazoiengatik ordu gehiago sartu behar izana gerta liteke. Horretaz gain, 
  lauhilekoan zehar bestelako ikasgaiek aurreikusitakoa baino ordu kopuru gehiago eskatzea ere gerta liteke.
  \item \textbf{Presioa:} Nahiz eta lan egin beharra dagoenean lan egiten duen eta denbora modu egokian kudeatzen duen pertsona bezala identifikatzen naizen,
  gerta liteke lauhileko honek suposatu dezakeen presioak eta lan-kargak nire errendimendua eragoztea.
  \item \textbf{Prokrastinazioa:} gailu elektronikoetara etengabeko esposizioak (bideojokoak, sare-sozialak, etab.) nire arreta eta errendimendua oztopa dezake.
  \item \textbf{Gaixotasunak:} aurreikusi ezin diren osasun-arazoek denbora-galera eragin dezakete, bai ikasketa-prozesuan, bai proiektuaren garapenean.
  \item \textbf{Kirolaren ondorioz sortzen diren lesioak:} astero 5 gimnasio entrenamendu egiteak, karga nahiko pisutsuak mugitzen ditudalarik, lesioak eragin ditzake, 
  proiektuaren garapena oztopa dezaketenak.
  \item \textbf{Berandu oheratzea:} egunean zeharreko jarduerak eta konpromisoak direla eta, gerta liteke aisialdirako denbora gutxi izatea eta denbora hori berandu oheratuz betetzea,
  eguneroko errendimenduan eragin handia izan dezakeena.
\end{itemize}

\subsection{Probabilitatea eta inpaktua}

Identifikatutako arrisku bakoitza lehentasunen arabera sailkatzeko, Probabilitate-Inpaktu matrizea erabili da. 
Metodologia honi esker, arrisku bakoitzari balio kuantitatibo bat esleitu zaio $P \times I$ formula erabiliz. 
Emaitzaren arabera, arriskuak hiru mailatan kategorizatu dira: kritikoak ($P \times I \ge 15$), esanguratsuak ($3 < P \times I \le 9$) eta moderatuak ($P \times I \le 3$).

\begin{table}[ht]
\centering
\begin{tabular}{|c|c|c|c|}
  \hline
  \textbf{P/I} & \textbf{Baxua (1)} & \textbf{Ertaina (3)} & \textbf{Altua (5)} \\ \hline
  \textbf{Gertagaitz (1)} & \cellcolor{green!30}\textbf{1} & \cellcolor{green!30}\textbf{3} & \cellcolor{yellow!50}\textbf{5}\\ \hline
  \textbf{Erdia (3)} & \cellcolor{green!30}\textbf{3} & \cellcolor{yellow!50}\textbf{9} & \cellcolor{red!40}\textbf{15} \\ \hline
  \textbf{Probablea (5)} & \cellcolor{yellow!50}\textbf{5} & \cellcolor{red!40}\textbf{15} & \cellcolor{red!40}\textbf{25} \\ \hline
\end{tabular}
\caption{Arrisku matrizea probabilitate eta inpaktuaren arabera}%
\label{tab:arrisku_matrizea}
\end{table}

\newpage
\subsection{Arrisku-adierazle nagusiak eta konponbide-adierazleak}

Arriskuak gauzatu aurretik detektatzeko, alerta-seinaleak edo adierazle nagusiak definitu dira. 
Adierazle hauek neurgarriak eta zehatzak dira, eta aukera ematen dute arrisku baten probabilitatea handitzen ari den jakiteko, horrela erantzun goiztiarra eman ahal izateko. 
Era berean, egoera normaltzen dela baieztatzeko konponbide-adierazleak ere zehaztu dira.

\begin{table}[H]
  \centering
  \renewcommand{\arraystretch}{1.3}
  \resizebox{\textwidth}{!}{%
  \begin{tabular}{|C{1.8cm}|C{2.1cm}|C{1.4cm}|C{1cm}|C{1.4cm}|C{1.8cm}|C{2.2cm}|C{2cm}|}
    \hline
    \textbf{Arriskuaren IDa} & \textbf{Arriskuaren deskribapena} & \textbf{$P$} & \textbf{$I$} & \textbf{Arriskua ($ P \times I $)} & \textbf{Kategoria} & \textbf{Arrisku adierazle nagusiak} & \textbf{Konponbide adierazleak} \\ \hline
    R-1 & Ustekabeko lan-karga & Probablea & Ertaina & 15 & Kritikoa & \raggedright-Ataza sinpleetan esperotako baino denbora gehiago inbertitu behar izana  & \raggedright\arraybackslash-Bestelako jardueretarako denbora soberan izatea \\ \hline
    R-2 & Presioa & Erdia & Ertaina & 9 & Esanguratsua & \raggedright-Urduritasuna \newline -Humore txarra \newline -Azukrearekiko tentazio handiagoa & \raggedright\arraybackslash-Humore ona \newline-Lasaitasuna egunean zehar \\ \hline
    R-3 & Prokrastinazioa & Probablea & Ertaina & 15 & Kritikoa & \raggedright-Gailu elektronikoak erabiltzeko tentazio ugariagoak & \raggedright\arraybackslash-Egunean zehar proposatutako betebeharrak betetzea oztoporik gabe \\ \hline
    R-4 & Gaixotasunak & Gertagaitz & Ertaina & 3 & Moderatua & \raggedright-Gorputzaren nekea eta ahultasuna \newline -Buruko mina \newline -Eztula\newline -Mukiak & \raggedright\arraybackslash-Egunean zehar energia altua izatea \newline-Ondoeza desagertzea\\ \hline
    R-5 & Kirolaren ondorioz sortzen diren lesioak & Gertagaitz & Ertaina & 3 & Moderatua & \raggedright-Giharren ondoeza pisuak mugitzerakoan \newline -Mugikortasun murrizketak & \raggedright\arraybackslash-Gimnasioan hobekuntzak ikustea (kargak, errepikapenak, etab.) \newline-Giharren ondoezarik ez \\ \hline
    R-6 & Berandu oheratzea & Erdia & Ertaina & 9 & Esanguratsua & \raggedright-Etxera berandu iristea, aisialdirako denborarik izan gabe & \raggedright\arraybackslash-Egunero gogotsu oheratzea \newline-Oheratu baino lehen gailu elektronikoak erabili ez izana \\ \hline
  \end{tabular}}
  \caption{Identifikatutako arriskuen informazioa}%
  \label{tab:arriskuen_informazioa}
\end{table}

\newpage
\section{Hirugarren fasea: Arriskuen erantzun plana}

Arriskuen kudeaketa eraginkorra ez da soilik identifikazioan oinarritzen, baizik eta horiei aurre egiteko estrategia zehatzak prestatzean. 
Fase honetan, arrisku bakoitzerako bi plan mota garatu dira: prebentzio-plana eta kontingentzia-plana. 

\noindent Prebentzio-planaren helburua arriskua gertatu aurretik esku hartzea da, 
haren gertatzeko probabilitatea murrizteko edota inpaktua txikitzeko. Neurri hauek proiektuaren egunerokoan integratu dira ezustekoak saihesteko.

\noindent Kontingentzia-planean, ordea, jada gertatutako arriskuek sor ditzaketen kalteak minimizatzea eta egoera berreskuratzeko ekintzak zehaztea bilatzen da. 
Neurri hauek behin prebentzio-neurriek huts egiten dutenean aplikatzen dira.

\begin{table}[H]
  \centering
  \renewcommand{\arraystretch}{1.3}
  \resizebox{\textwidth}{!}{%
  \begin{tabular}{|C{0.6cm}|C{6cm}|C{6cm}|}
    \hline
    \textbf{ID} & \textbf{Prebentzio-plana} & \textbf{Kontingentzia-plana} \\ \hline
    R-1 & \raggedright\arraybackslash- Praktiketako arduradunarekin aste bakoitzeko helburuak eta ordutegia (15 ordu) zorrotz zehaztu. \newline- Astero irakasgai guztien lan-karga berrikusi eta lehentasunak ezarri. \newline- Segurtasun-tarteak utzi asteko plangintzan ezustekoetarako. & \raggedright\arraybackslash- Aisialdiko edo kirol-jarduerak aldi baterako murriztu denbora irabazteko. \newline- Ezinbestekoak ez diren entregak edo konpromiso sozialak atzeratu. \newline- Pilatutako lana berreskuratzeko beranduago oheratu. \\ \hline
    R-2 & \raggedright\arraybackslash- Egunero atsedenaldi laburrak (\emph{Pomodoro} teknika, adibidez) eta deskonexio momentuak errespetatu. \newline- Helburu errealistak eta zatikatuak jarri egun bakoitzerako, gauzak pixkanaka lantzeko. & \raggedright\arraybackslash- Egun libre oso bat hartu, erabat deskonektatzeko eta energia berreskuratzeko. \newline- Ingurukoei (lagunak, familia \ldots) laguntza eskatu eta egoera pertsonala haiekin partekatu, presioak eragindako sama arintzeko. \\ \hline
    R-3 & \raggedright\arraybackslash- Ikasketa prozesuetan mugikorra beste gela batean utzi. \newline- ``Forest'' edo antzeko aplikazioak erabili distrakzio digitalak blokeatzeko. \newline- Lan-saioak gailu elektronikorik gabeko ingurune batean (adibidez, liburutegian) egin, distrakzioak ekiditeko. & \raggedright\arraybackslash- ``Detox digital'' asteburu bat egin: sare sozialak eta bideojokoak erabat kendu. \newline- Elkarrekin ikasteko eta lan egiteko kide bat bilatu, erantzukizunak betetzera bultzatu dezakeena. \\ \hline
    R-4 & \raggedright\arraybackslash- Lo orduak (7-8 ordu) eta elikadura osasuntsua zorrotz zaindu. \newline- Higiene neurriak mantendu eta tenperatura aldaketetatik babestu. & \raggedright\arraybackslash- Gorputzari arreta eman eta eskatzen duen atsedena behar den bezala hartu. \newline- Ikaskideei galdutako klaseetako apunteak eta materiala eskatu. \newline- Osatu ondoren, ikasketa-plana berrantolatu galdutakoa berreskuratzeko. \\ \hline
    R-5 & \raggedright\arraybackslash- Entrenamenduetan teknika lehenetsi pisuaren gainetik. \newline- Beroketa eta luzatze ariketak beti egin saio bakoitzaren aurretik eta ondoren. & \raggedright\arraybackslash- Fisioterapeuta batengana joan lesioa tratatzeko. \newline- Entrenamenduak aldi baterako eten eta denbora hori errehabilitaziorako eta ikasketak aurreratzeko erabili. \\ \hline
    R-6 & \raggedright\arraybackslash- Oheratzea gogorarazteko alarmak ezarri. \newline- Oheratu baino ordubete lehenago pantailak saihestu. \newline- Afari arinak eta goiztiarrak egin. & \raggedright\arraybackslash- Hurrengo egunean 20 minutuko (luzeagoa ez) siesta egin energia berreskuratzeko. \newline- Hurrengo gauean ohikoa baino ordubete lehenago oheratu lo-zorra ezabatzeko. \newline- Goizeko kafea edo bestelako estimulatzaileak ekiditu eguerditik aurrera, lo-zikloa berrezartzeko. \\ \hline
  \end{tabular}}
  \caption{Arriskuen gaineko erantzun planak}%
  \label{tab:erantzun_planak}
\end{table}

\end{document}