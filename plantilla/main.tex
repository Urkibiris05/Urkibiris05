\documentclass{article}

%%\chapterstyle{veelo}
%% KAPITULU FORMATU BATZUK ARAZOAK EMATEN DITUZTE DOKUMENTUA 
%% MOZTERAKOAN (croppedsize AUKERAREKIN, BEHEAN AGERTZEN DENA). 
%% JARRAIAN ALTERNATIBA SGURU BATZUK DAUDE

%\chapterstyle{default}
%\chapterstyle{bianchi}
%\chapterstyle{brotherton}
%\chapterstyle{demo2}
%\chapterstyle{ell}
%\chapterstyle{ger}
%\chapterstyle{wilsondob}






%% PAKETE HONEN AGINDUAK ERABILTZEN BADIRA, ALDATU HIZKUNTZA AUKERA %%
%% SI SE USAN LOS COMANDOS DE ESTE PAQUETE, CAMBIAR LA OPCIÓN DE IDIOMA %%
\usepackage{cancel}
\usepackage{amsmath}
\usepackage[utf8]{inputenc}
\usepackage{amssymb}
\usepackage{tikz}
\usetikzlibrary{positioning}
\usetikzlibrary{arrows.meta}
\usepackage{enumitem}
\usepackage{graphicx}
\usepackage[linkcolor=black, citecolor=black, urlcolor=black]{hyperref}
\usepackage[letterpaper,top=2cm,bottom=2cm,left=3cm,right=3cm,marginparwidth=1.75cm]{geometry}

%% INFORMAZIOA / INFORMACIÓN %%
\newcommand{\egilea}{
    Urko Bidaurre\\
}
\newcommand{\irakaslea} {
      Alicia Perez Ramirez
}
\newcommand{\izenburua}{Datuen deskribapen operatiboa}
\newcommand{\data}{\today}


%% BAKARRIK BAT DESKOMENTATU, GRADU AMAIERAKO LANA BADA. BESTELA, DENAK KOMENTATU%%
%% DESCOMENTAR SOLO UNO SI ES UN TRABAJO FIN DE GRADO. SINO, COMENTAR TODOS%%

\newcommand{\ikasketak}{Kudeaketaren eta Informazio Sistemen Informatikaren Ingeniaritzako Gradua}
%\newcommand{\ikasketak}{Grado en Ingeniería Informática}
%\newcommand{\ikasketak}{Adimen Artifizialeko Gradua}
%\newcommand{\ikasketak}{Grado en Inteligencia Artificial}


%% BAKARRIK BAT DESKOMENTATU, INGENIARITZA INFORMATIKAKOA BADA. BESTELA, DENAK KOMENTATU %%
%% DESCOMENTAR SOLO UNO SI ES EN INGENIERÍA INFORMÁTICA. SINO, COMENTAR TODOS %%

%\newcommand{\espezialitatea}{Konputazioa}
%\newcommand{\espezialitatea}{Computación}
%\newcommand{\espezialitatea}{Software Ingeniaritza}
%\newcommand{\espezialitatea}{Ingeniería de Software}
%\newcommand{\espezialitatea}{Konputagailuen Ingeniaritza}
\newcommand{\espezialitatea}{
    Erabakiak Hartzeko Euskarri Sistemak\\
    }


%% BAKARRIK BAT DESKOMENTATU MASTER AMAIERAKO LANA BADA. BESTELA, DENAK KOMENTATU%%
%% DESCOMENTAR SOLO UNO SI ES UN TRABAJO FIN DE MASTER. SINO, COMENTAR TODOS%%

%\newcommand{\ikasketak}{Máster Universitario en Ingeniería Computacional y Sistemas Inteligentes}
%\newcommand{\ikasketak}{Konputazio Ingeniaritza eta Sistema Adimentsuak Unibertsitate Masterra}

%% TESIENTZAKO KONFIGURATU
%\newcommand{\saila}{Department of Computer Science and Artificial Intelligence}
%\newcommand{\phdarloa}{Computer Science}
%% KOMENTATU A4 FORMATUAN ATERATZEKO, FORMATU TXIKIAGOAN AGERTZEKO 
%% GERO INPRIMATZEKO. MOZTEKO MARKAK IKUSTEKO GEHITU showtrims AUKERA
%% memoir KLASEAREN ARGUMENTUETAN
%\newcommand{\croppedsize}

%% DOKUMENTUAREN HIZKUNTZA / IDIOMA DEL DOCUMENTO %%
%% BAKARRIK BAT DESKOMENTATU / DESCOMENTAR SOLO UNO %%
\newcommand{\euskaraz}
%\newcommand{\castellano}
%\newcommand{\english}


\input config/macros

\title{\izenburua}
\author{\egilea}
\date{\data}

\begin{document}
\thispagestyle{empty}

\newcommand{\HRule}{\rule{\linewidth}{0.5mm}} 

\thispagestyle{empty}
\begin{center}
	
  \includegraphics[width=0.75\textwidth]{config/FacultadInformatica-Gipuzkoa-bilingue-positivo-alta.jpg} \\[2cm]
  
  
  {\LARGE {\gapizenburua}}\\[0.5cm]
  {\Large \ikasketak}\\[0.25cm]
  {\espezialitatea}
  \vspace{2cm}
  
  
  \HRule \\[0.5cm]
  {\LARGE 
    \textbf{\izenburua}
  }
  \HRule \\[0.5cm]

  {\Large \textsl{\egilea}
  
   
  \vfill
  
  \textbf{\zuzendariaktestua}\\
  \irakaslea\\[2cm]
  
  \data
  }
  
\end{center}
\newpage
\tableofcontents
\newpage
\section{Materiala}
\begin{itemize}
  \item Weka aplikazioa
  \item Baliabide bibliografikoak:
  \begin{itemize}
    \item Informazio orokorra adibideekin: [Witten et al., 2011, Chap. 2]
    \item Kontsulta praktikoak: \url{https://waikato.github.io/weka-wiki/}
  \end{itemize}
  \item eGelatik eskuragarri:
  \begin{itemize}
    \item Baliabide orokorrak: aplikazioaren eskuliburua
    \item Praktikarako datu-sorta: heart-c.arff
  \end{itemize}
\end{itemize}
\section{Problemaren definizioa eta aplikazioak}

\subsection{Helburuak}
\emph{Cursiva}

\noindent \textbf{Negrita}
 
\begin{align*}
\min \quad & Z = \sum_{i,j} c_{ij} x_{ij} \\
\text{non} \quad & \sum_j x_{ij} = 1 \quad \forall i 
\quad \textit{(hiri guztiek esleipen bakarra)} \\
& \sum_i x_{ij} = 1 \quad \forall j 
\quad \textit{(hiri bakoitza behin esleitua)} \\
& x_{ij} \in \{0,1\} \quad \forall i,j
\end{align*}

\[
\renewcommand{\arraystretch}{1.2}
\begin{array}{c|cccc|c}
    & 1 & 2 & 3 & \cdots & n \\
    \hline
  1 & 0 & d_{12} & d_{13} & \cdots & d_{1n} \\
  2 & d_{21} & 0 & d_{23} & \cdots & d_{2n} \\
  3 & d_{31} & d_{32} & 0 & \cdots & d_{3n} \\
  \vdots & \vdots & \vdots & \vdots & \ddots & \vdots \\
  n & d_{n1} & d_{n2} & d_{n3} & \cdots & 0
\end{array}
\]

\newpage
\bibliographystyle{alpha}
\bibliography{sample}
\begin{itemize}
    \item \emph{https://www.lystloc.com/blog/what-is-a-travelling-salesman-problem-tsp/}
    \item \emph{SYMPLEX: SIMPLEX SOLVER}
\end{itemize}

\end{document}